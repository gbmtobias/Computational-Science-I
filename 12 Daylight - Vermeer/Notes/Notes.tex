%% Basierend auf einer TeXnicCenter-Vorlage von Mark Müller
%%%%%%%%%%%%%%%%%%%%%%%%%%%%%%%%%%%%%%%%%%%%%%%%%%%%%%%%%%%%%%%%%%%%%%%

% Wählen Sie die Optionen aus, indem Sie % vor der Option entfernen  
% Dokumentation des KOMA-Script-Packets: scrguide

%%%%%%%%%%%%%%%%%%%%%%%%%%%%%%%%%%%%%%%%%%%%%%%%%%%%%%%%%%%%%%%%%%%%%%%
%% Optionen zum Layout des Artikels                                  %%
%%%%%%%%%%%%%%%%%%%%%%%%%%%%%%%%%%%%%%%%%%%%%%%%%%%%%%%%%%%%%%%%%%%%%%%
\documentclass[%
%a5paper,							% alle weiteren Papierformat einstellbar
%landscape,						% Querformat
%10pt,								% Schriftgröße (12pt, 11pt (Standard))
%BCOR1cm,							% Bindekorrektur, bspw. 1 cm
%DIVcalc,							% führt die Satzspiegelberechnung neu aus
%											  s. scrguide 2.4
%twoside,							% Doppelseiten
%twocolumn,						% zweispaltiger Satz
%halfparskip*,				% Absatzformatierung s. scrguide 3.1
%headsepline,					% Trennline zum Seitenkopf	
%footsepline,					% Trennline zum Seitenfuß
%titlepage,						% Titelei auf eigener Seite
%normalheadings,			% Überschriften etwas kleiner (smallheadings)
%idxtotoc,						% Index im Inhaltsverzeichnis
%liststotoc,					% Abb.- und Tab.verzeichnis im Inhalt
%bibtotoc,						% Literaturverzeichnis im Inhalt
%abstracton,					% Überschrift über der Zusammenfassung an	
%leqno,   						% Nummerierung von Gleichungen links
%fleqn,								% Ausgabe von Gleichungen linksbündig
%draft								% überlangen Zeilen in Ausgabe gekennzeichnet
]
{scrartcl}

%\pagestyle{empty}		% keine Kopf und Fußzeile (k. Seitenzahl)
%\pagestyle{headings}	% lebender Kolumnentitel

%% Deutsche Anpassungen %%%%%%%%%%%%%%%%%%%%%%%%%%%%%%%%%%%%%
\usepackage[english]{babel}
\usepackage[T1]{fontenc}
\usepackage[utf8]{inputenc}

\usepackage{amsthm} % Theorem-Packet
\usepackage{amsmath}
\usepackage{amssymb}

\usepackage{stmaryrd} % Blitzsymbol
\usepackage{fancyhdr} % Für Kopfzeile
\usepackage{graphicx} % Einbinden von Grafiken
\usepackage{bbding} % Für das Häckchen
\usepackage{amscd} % Kommutative Diagramme
\usepackage{mathtools} % Für das Definitionssymbol

\usepackage{listings}
\usepackage{courier}

\pagestyle{fancy}
\lhead{QGD411}\chead{Exercise notes: Daylight \& Vermeer}\rhead{HS 2013} % Kopfzeile

\newtheoremstyle{plain}%  name
  {.5\baselineskip}% Space above
  {.5\baselineskip}% Space below
  {}% Body font
  {}% Indent amount (empty = no indent, \parindent = para indent)
  {\bfseries}% Thm head font
  {:}% Punctuation after thm head
  { }% Space after thm head: " " = normal interword space; \newline = linebreak
  {}% Thm head spec (can be left empty, meaning `normal')
  
\makeatletter % Matrizen mit opitonalen Linien
\renewcommand*\env@matrix[1][*\c@MaxMatrixCols c]{%
  \hskip -\arraycolsep
  \let\@ifnextchar\new@ifnextchar
  \array{#1}}
\makeatother

\theoremstyle{plain}
\newtheorem*{bsp}{Beispiel} % Beispiele ohne Nummerierung
\newtheorem*{bws}{Beweis} % Beweise ohne Nummerierung 
\newenvironment{beweis}{\begin{bws}~\vspace{0.5\baselineskip}}{\hfill $\qedsymbol$\end{bws}}
\newenvironment{beispiel}{\begin{bsp}~\vspace{0.5\baselineskip}}{\end{bsp}}

\usepackage{lmodern} % Type1-Schriftart für nicht-englische Texte

\usepackage{enumerate}

\renewcommand\theenumi{\roman{enumi}}
\renewcommand\labelenumi{\theenumi)}

%% Packages für Grafiken & Abbildungen %%%%%%%%%%%%%%%%%%%%%%
%\usepackage{graphicx} %%Zum Laden von Grafiken
%\usepackage{subfig} %%Teilabbildungen in einer Abbildung
%\usepackage{tikz} %%Vektorgrafiken aus LaTeX heraus erstellen

%\setlength{\parindent}{0pt} % kein Einzug


%% Beachten Sie:
%% Die Einbindung einer Grafik erfolgt mit \includegraphics{Dateiname}
%% bzw. über den Dialog im Einfügen-Menü.
%% 
%% Im Modus "LaTeX => PDF" können Sie u.a. folgende Grafikformate verwenden:
%%   .jpg  .png  .pdf  .mps
%% 
%% In den Modi "LaTeX => DVI", "LaTeX => PS" und "LaTeX => PS => PDF"
%% können Sie u.a. folgende Grafikformate verwenden:
%%   .eps  .ps  .bmp  .pict  .pntg


%% Bibliographiestil %%%%%%%%%%%%%%%%%%%%%%%%%%%%%%%%%%%%%%%%%%%%%%%%%%
%\usepackage{natbib}

\begin{document}

\lstset{basicstyle=\ttfamily, breakatwhitespace=false, breaklines=true, frame=single, xleftmargin=\parindent, aboveskip=\baselineskip, belowskip=\baselineskip}

%\pagestyle{empty} %%Keine Kopf-/Fusszeilen auf den ersten Seiten.


%%%%%%%%%%%%%%%%%%%%%%%%%%%%%%%%%%%%%%%%%%%%%%%%%%%%%%%%%%%%%%%%%%%%%%%
%% Ihr Artikel                                                       %%
%%%%%%%%%%%%%%%%%%%%%%%%%%%%%%%%%%%%%%%%%%%%%%%%%%%%%%%%%%%%%%%%%%%%%%%

%% eigene Titelseitengestaltung %%%%%%%%%%%%%%%%%%%%%%%%%%%%%%%%%%%%%%%    
%\begin{titlepage}
%Einsetzen der TXC Vorlage "Deckblatt" möglich
%\end{titlepage}

%% Angaben zur Standardformatierung des Titels %%%%%%%%%%%%%%%%%%%%%%%%
\titlehead{\center{University of Zurich - HS 2013}}
%\subject{Typisierung}
\title{Computational Science I\\Exercise notes: Daylight \& Vermeer\\\rule{1.0\textwidth}{1.0pt}}
\author{Tobias Grubenmann}
%\and{Der Name des Co-Autoren}
%\thanks{Fußnote}			% entspr. \footnote im Fließtext
%\date{}							% falls anderes, als das aktuelle gewünscht
%\publishers{Herausgeber}

%% Widmungsseite %%%%%%%%%%%%%%%%%%%%%%%%%%%%%%%%%%%%%%%%%%%%%%%%%%%%%%
%\dedication{Widmung}

\maketitle 						% Titelei wird erzeugt

%% Zusammenfassung nach Titel, vor Inhaltsverzeichnis %%%%%%%%%%%%%%%%%
%\begin{abstract}
% Für eine kurze Zusammenfassung des folgenden Artikels.
% Für die Überschrift s. \documentclass[abstracton].
%\end{abstract}

%% Erzeugung von Verzeichnissen %%%%%%%%%%%%%%%%%%%%%%%%%%%%%%%%%%%%%%%
%\tableofcontents			% Inhaltsverzeichnis
%\listoftables				% Tabellenverzeichnis
%\listoffigures				% Abbildungsverzeichnis


%% Der Text %%%%%%%%%%%%%%%%%%%%%%%%%%%%%%%%%%%%%%%%%%%%%%%%%%%%%%%%%%%

\section*{Exercise 1}

The following picture shows the daylight intensity for the 8th December 2013 at 16:35 CET (sunset at Zurich). 

\begin{center}
\centering
\includegraphics[width=0.6\linewidth]{../Daylight.png}
\captionof{figure}{Daylight on 8th December 2013 at 16:35 CET}
\end{center}

\section*{Exercise 2}

I've chosen a value of $71^{\circ}$ instead of $90^{\circ}$ for the angle of the window with respect to the camera since I've got better results with this assumption.

For the four parameters (3D translation and viewer distance from the pinhole camera) i got the values $(1.53151851, 8.20440454,   62.5873597, 15119.0110)$

The following picture shows the reprojection of the window using the above parameters:

\begin{center}
\centering
\includegraphics[width=0.6\linewidth]{../resultSmall.png}
\captionof{figure}{Reprojection of the window}
\end{center}

%% Bibliographie unter Verwendung von dinnat %%%%%%%%%%%%%%%%%%%%%%%%%%
%\setbibpreamble{Präambel}		% Text vor dem Verzeichnis
%\bibliographystyle{dinat}
%\bibliography{bibliographie}	% Sie benötigen einen *.bib-Datei

\end{document}
